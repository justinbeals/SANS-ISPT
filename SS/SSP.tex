\chapter{Server Security Policy}
\CommonIntroduction
\section{Overview}
Unsecured and vulnerable servers continue to be a major entry point for malicious threat actors.  
Consistent server installation policies, ownership and configuration management are all about doing the basics well. 
\section{Purpose}
The purpose of this policy is to establish standards for the base configuration of internal server equipment that is owned and/or operated by \CompanyName{}.  
Effective implementation of this policy will minimize unauthorized access to \CompanyName{} proprietary information and technology.
\section{Scope}
All employees, contractors, consultants, temporary\oxford{} and other workers at \oldnew{Cisco}{\CompanyName} and its subsidiaries must adhere to this policy.  
This policy applies to server equipment that is owned, operated\oxford{} or leased by \oldnew{Cisco}{\CompanyName} or registered under a \oldnew{Cisco}{\CompanyName}-owned internal network domain. 
This policy specifies requirements for equipment on the internal \oldnew{Cisco}{\CompanyName} network.  
\del{For secure configuration of equipment external to \oldnew{Cisco}{\CompanyName} on the \gls{dmz}, see the \textit{Internet DMZ Equipment Policy.}}%TODO DMZ Policy (DMZ Lab Security Policy is old/retired
\section{Policy}
\subsection{General Requirements}
\begin{enumerate}
\item
All internal servers deployed at \CompanyName{} must be owned by an operational group that is responsible for system administration.  
Approved server configuration guides must be established and maintained by each operational group, based on business needs and approved by \gls{infosec}.  
Operational groups should monitor configuration compliance and implement an exception policy tailored to their environment.  
Each operational group must establish a process for changing the configuration guides, which includes review and approval by \gls{infosec}.  
The following items must be met:
\begin{itemize}
\item
Servers must be registered within the corporate enterprise management system.  
At a minimum, the following information is required to positively identify the \gls{ptoc}: 
\begin{itemize}
\item
Server contact(s) and location, and a backup contact 
\item
Hardware \ins{make and model, }and \oldnew{O}{o}perating \oldnew{S}{s}ystem\oldnew{/}{ }\oldnew{V}{v}ersion 
\item
Main functions and applications, if applicable 
\end{itemize}
\item
Information in the corporate enterprise management system must be kept up-to-date. 
\item
Configuration changes for production servers must follow the appropriate change management procedures
\end{itemize}
\item
For security, compliance\oxford{} and maintenance purposes, authorized personnel may monitor and audit equipment, systems, processes\oxford{} and network traffic per the Audit Policy.%TODO Audit Policy
\end{enumerate}
\subsection{Configuration Requirements}
\begin{enumerate}
\item
Operating \oldnew{S}{s}ystem configuration should be in accordance with approved \gls{infosec} guidelines. 
\item
Services and applications that will not be used must be disabled where practical.
\item
Access to services should be logged and/or protected through access-control methods such as a web application firewall, if possible. 
\item
The most recent security patches must be installed on the system as soon as practical, the only exception being when immediate application would interfere with business requirements. 
\item
Trust relationships between systems are a security risk, and their use should be avoided.  
Do not use a trust relationship when some other method of communication is sufficient. 
\item
Always use standard security principles of least required access to perform a function.  
Do not use \texttt{root} or \texttt{Administrator} when a non-privileged account will do. 
\item
If a methodology for secure channel connection is available (\ie, technically feasible), privileged access must be performed over secure channels, (\eg, encrypted network connections using \gls{ssh} or \gls{ipsec}). 
\item
Servers should be physically located in an access-controlled environment. 
\item
Servers are specifically prohibited from operating from uncontrolled cubicle areas. 
\end{enumerate}
\subsection{Monitoring}
\begin{enumerate}
\item
All security-related events on critical or sensitive systems must be logged and audit trails saved as follows: 
\begin{itemize}
\item
All security related logs will be kept online for a minimum of 1 week. 
\item
Daily incremental tape backups will be retained for at least 1 month. 
\item
Weekly full tape backups of logs will be retained for at least 1 month. 
\item
Monthly full backups will be retained for a minimum of 2 years. 
\end{itemize}
\item 
Security-related events will be reported to \gls{infosec}, who will review logs and report incidents to \gls{it} management.  
Corrective measures will be prescribed as needed.  
Security-related events include, but are not limited to: 
\begin{itemize}
\item
Port-scan attacks 
\item
Evidence of unauthorized access to privileged accounts 
\item
Anomalous occurrences that are not related to specific applications on the host. 
\end{itemize}
\end{enumerate}
\CommonPolicyCompliance
\section{Related Standards, Policies\oxford{} and Processes}
\begin{itemize}
\item
Audit Policy%TODO
\item
DMZ Equipment Policy%TODO
\end{itemize}
\CommonDefinitionsAndTerms
\begin{itemize}
\item \gls{dmz}
\end{itemize}
\CommonRevisionHistory
