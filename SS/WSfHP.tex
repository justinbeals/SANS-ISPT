\chapter{Workstation Security (for HIPAA) Policy}
\section{Overview}
See \see{SS:WSfHP:Pu}.
\section{Purpose}\label{SS:WSfHP:Pu}
The purpose of this policy is to provide guidance for workstation security for \CompanyName{} workstations in order to ensure the security of information on the workstation and information the workstation may have access to.  
Additionally, the policy provides guidance to ensure the requirements of the HIPAA Security Rule \q{Workstation Security} Standard 164.310(c) are met.
\section{Scope}
This policy applies to all \CompanyName{} employees, contractors, workforce members, vendors\oxford{} and agents with a \CompanyName{}-owned or personal-workstation connected to the \CompanyName{} network.
\section{Policy}
Appropriate measures must be taken when using workstations to ensure the confidentiality, integrity and availability of sensitive information, including \gls{phi} and that access to sensitive information is restricted to authorized users.  
\begin{enumerate}
\item
Workforce members using workstations shall consider the sensitivity of the information, including \gls{phi} that may be accessed and minimize the possibility of unauthorized access.
\item
\CompanyName{} will implement physical and technical safeguards for all workstations that access electronic protected health information to restrict access to authorized users. 
\item
Appropriate measures include: 
\begin{itemize}
\item
Restricting physical access to workstations to only authorized personnel.
\item
Securing workstations (screen lock or logout) prior to leaving area to prevent unauthorized access.
\item
Enabling a password-protected screen saver with a short timeout period to ensure that workstations that were left unsecured will be protected.  The password must comply with \CompanyName{} \PasswordPolicies{}. 
\item
Complying with all applicable password policies and procedures. See \CompanyName{} \PasswordPolicies{}. 
\item
Ensuring workstations are used for authorized business purposes only.
\item
Never installing unauthorized software on workstations.
\item
Storing all sensitive information, including \gls{phi} on network servers  
\item
Keeping food and drink away from workstations in order to avoid accidental spills.
\item
Securing laptops that contain sensitive information by using cable locks or locking laptops up in drawers or cabinets. 
\item
Complying with the Portable Workstation Encryption Policy%TODO
\item
Complying with the Baseline Workstation Configuration Standard%TODO
\item
Installing privacy screen filters or using other physical barriers to alleviate exposing data. 
\item
Ensuring workstations are left on but logged off in order to facilitate after-hours updates.
\item
Exit running applications and close open documents
\item
Ensuring that all workstations use a surge protector (not just a power strip) or a \gls{ups} (battery backup).
\item
If wireless network access is used, ensure access is secure by following \oldnew{the Wireless Communication policy}{\see{NS:WCP}}
\end{itemize}
\end{enumerate}
\CommonPolicyCompliance
\section{Related Standards, Policies\oxford{} and Processes}
\begin{itemize}
\item
\see{G:PCG}
\item
\see{G:PPP}
\item
Portable Workstation Encryption Policy
\item
\see{NS:WCP}
\item
\see{NS:WCS}
\item
\href{http://www.hipaasurvivalguide.com/hipaa-regulations/164-310.php}{HIPPA 164.210}\newline\url{http://www.hipaasurvivalguide.com/hipaa-regulations/164-310.php}
\item
\href{http://abouthipaa.com/about-hipaa/hipaa-hitech-resources/hipaa-security-final-rule/164-308a1i-administrative-safeguards-standard-security-management-process-5-3-2-2/}{About HIPPA}\newline\url{http://abouthipaa.com/about-hipaa/hipaa-hitech-resources/hipaa-security-final-rule/164-308a1i-administrative-safeguards-standard-security-management-process-5-3-2-2/}
\end{itemize}
\section{Definitions and Terms}
None.
\CommonRevisionHistory