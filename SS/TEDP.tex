\chapter{Technology Equipment Disposal Policy}
\CommonIntroduction
\section{Overview}
Technology equipment often contains parts which cannot simply be thrown away.  Proper disposal of equipment is both environmentally responsible and often required by law.  
In addition, hard drives, USB drives, \cdrom{}s and other storage media contain various kinds of \CompanyName{} data, some of which is considered sensitive.  
In order to protect our constituent's data, all storage mediums must be properly erased before being disposed of.  
However, simply deleting or even formatting data is not considered sufficient.  
When deleting files or formatting a device, data is marked for deletion, but \ins{rather it }is still accessible until \del{being }overwritten by a new file.  
Therefore, special tools must be used to securely erase data prior to equipment disposal.  
\section{Purpose}
The purpose of this policy it to define the guidelines for the disposal of technology equipment and components owned by \CompanyName{}. 
\section{Scope}
This policy applies to any computer/technology equipment or peripheral devices that are no longer needed within \CompanyName{} including, but not limited to the following:  personal computers, servers, hard drives, laptops, mainframes, \smartphone{}s, or \handheld{} computers (\ie, Windows Mobile, iOS\oxford{} or Android-based devices), peripherals (\ie, keyboards, mice, speakers), printers, scanners, typewriters, compact and floppy discs, portable storage devices (\ie, USB drives), backup tapes, printed materials\ins{, \etc}. 

All \CompanyName{} employees and affiliates must comply with this policy. 
\section{Policy}
\subsection{Technology Equipment Disposal}
\begin{enumerate}
\item
When \oldnew{T}{t}echnology assets have reached the end of their useful life they should be sent to the \EquipmentDisposalTeam{} office for proper disposal.
\item
The \EquipmentDisposalTeam{} will securely erase all storage mediums in accordance with current industry best practices.
\item
All data including, all files and licensed software shall be removed from equipment using disk sanitizing software that cleans the media overwriting each and every disk sector of the machine with zero-filled blocks, meeting Department of Defense standards.
\item
No computer or technology equipment may be sold to any individual other than through the processes identified in \oldnew{this policy  (Section 4.2 below)}{\see{SS:TEDP:Po:EPDE}}.
\item
No computer equipment should be disposed of via skips, dumps, landfill\ins{,} \etc.  
Electronic recycling bins may be periodically placed in locations around \CompanyName{}.  
These can be used to dispose of equipment.  
The \EquipmentDisposalTeam will properly remove all data prior to final disposal. 
\item
All electronic drives must be degaussed or overwritten with a commercially available disk cleaning program.  
Hard drives may also be removed and rendered unreadable (drilling, crushing\oxford{} or other demolition methods).
\item
Computer Equipment refers to desktop, laptop, tablet or netbook computers, printers, copiers, monitors, servers, \handheld{} devices, telephones, \cellphone{}s, disc drives or any storage device, network switches, routers, wireless access points, batteries, backup tapes, \etc.
\item
The \EquipmentDisposalTeam will place a sticker on the equipment case indicating \oldnew{the disk wipe has been performed}{that data has been successfully wiped from the device}.  
The sticker will include the date and the initials of the technician who performed the \oldnew{disk}{data} wipe.
\item
Technology equipment with non-functioning memory or storage technology will have the memory or storage device removed and it will be physically destroyed.
\end{enumerate}
\subsection{Employee Purchase of Disposed Equipment}\label{SS:TEDP:Po:EPDE}
\begin{enumerate}
\item
Equipment which is working, but reached the end of its useful life to \CompanyName{}, will be made available for purchase by employees.
\item
A lottery system will be used to determine who has the opportunity to purchase available equipment.
\item
All equipment purchases must go through the lottery process.  
Employees cannot purchase their office computer directly or \q{reserve} a system.  
This ensures that all employees have an equal chance of obtaining equipment.
\item
Finance and \gls{it} will determine an appropriate cost for each item.  
\item
All purchases are final.  
No warranty or support will be provided with any equipment sold.  
\item
Any equipment not in working order or remaining from the lottery process will be donated or disposed of according to current environmental guidelines.  
\item
\gls{it} has contracted with several organizations to donate or properly dispose of outdated technology assets.   
\item
Prior to leaving \CompanyName{} premises, all equipment must be removed from the \gls{it} inventory system.  
\end{enumerate}
\section{Policy Compliance}
\CommonPolicyCompliance
\section{Related Standards, Policies\oxford{} and Processes}
None.
\section{Definitions and Terms}
None.
\CommonRevisionHistory