\chapter{Information Logging Standard}
\CommonIntroduction
\section{Overview}
Logging from critical systems, applications and services can provide key information and potential indicators of compromise.  
Although logging information may not be viewed on a daily basis, it is critical to have from a forensics standpoint. 
\section{Purpose}
The purpose of this document attempts to address this issue by identifying specific requirements that information systems must meet in order to generate appropriate audit logs and integrate with an enterprise's log management function.  
The intention is that this language can easily be adapted for use in enterprise \gls{it} security policies and standards, and also in enterprise procurement standards and \gls{rfp} templates.  
In this way, organizations can ensure that new \gls{it} systems, whether developed in-house or procured, support necessary audit logging and log management functions.
\section{Scope}
This policy applies to all production systems on \CompanyName{} Network. 
\section{Standard}
\subsection{General Requirements}
All systems that handle confidential information, accept network connections, or make access control (authentication and authorization) decisions shall record and retain audit-logging information sufficient to answer the following questions:
\begin{enumerate}
\item
What activity was performed?
\item
Who or what performed the activity, including where or on what system the activity was performed from (subject)?
\item
What the activity was performed on (object)?
\item
When was the activity performed?
\item
What tool(s) was the activity was performed with?
\item
What was the status (such as success \vs{} failure), outcome\oxford{} or result of the activity?
\end{enumerate}
\subsection{Activities to be Logged}
Therefore, logs shall be created whenever any of the following activities are requested to be performed by the system:
\begin{enumerate}
\item\label{SS:ILS:S:AtbL:1}
Create, read, update\oxford{} or delete confidential information, including confidential authentication information such as passwords;
\item\label{SS:ILS:S:AtbL:2}
Create, update\oxford{} or delete information not covered in \see{SS:ILS:S:AtbL:1};%TODO Better Enumeration
\item
Initiate a network connection;
\item
Accept a network connection;
\item
User authentication and authorization for activities covered in \see{SS:ILS:S:AtbL:1} or \see{SS:ILS:S:AtbL:2} such as user login and logout;%TODO Better Enumeration
\item
Grant, modify\oxford{} or revoke access rights, including adding a new user or group, changing user privilege levels, changing file permissions, changing database object permissions, changing firewall rules\oxford{} and user password changes;
\item
System, network\oxford{} or services configuration changes, including installation of software patches\oldnew{ and}{,} updates\oldnew{,}{\oxford} or other installed software changes;
\item
Application process startup, shutdown\oxford{} or restart;
\item
Application process abort, failure\oxford{} or abnormal end, especially due to resource exhaustion or reaching a resource limit or threshold (such as for \gls{cpu}, memory, network connections, network bandwidth, disk space\oxford{} or other resources), the failure of network services such as \gls{dhcp} or \gls{dns}, or hardware fault; and
\item
Detection of suspicious/malicious activity such as from an \gls{ids} or \gls{ids}, \gls{av}, or anti-spyware system.
\end{enumerate}
\subsection{Elements of the Log}
Such logs shall identify or contain at least the following elements, directly or indirectly.  
In this context, the term \q{indirectly} means unambiguously inferred.
\begin{enumerate}
\item
Type of action\newline{}
Examples include authorize, create, read, update, delete\oxford{} and accept network connection.
\item
Subsystem performing the action\newline{}
Examples include process or transaction name, process\oxford{} or transaction identifier.
\item
Identifiers (as many as available) for the subject requesting the action\newline{}
Examples include user name, computer name, \gls{ip} address\oxford{} and \gls{mac} address.  
Note that such identifiers should be standardized in order to facilitate log correlation.
\item
Identifiers (as many as available) for the object the action was performed on\newline{}
Examples include file names accessed, unique identifiers of records accessed in a database, query parameters used to determine records accessed in a database, computer name, \gls{ip} address\oxford{} and \gls{mac} address.  
Note that such identifiers should be standardized in order to facilitate log correlation.
\item
Before and after values when action involves updating a data element, if feasible.
\item
Date and time the action was performed, including relevant \timezone{} information if not in \gls{utc}.
\item
Whether the action was allowed or denied by access-control mechanisms.
\item
Description and/or reason-codes of why the action was denied by the access-control mechanism, if applicable.
\end{enumerate}
\subsection{Formatting and Storage}
The system shall support the formatting and storage of audit logs in such a way as to ensure the integrity of the logs and to support enterprise-level analysis and reporting.  
Note that the construction of an actual enterprise-level log management mechanism is outside the scope of this document.  
Mechanisms known to support these goals include but are not limited to the following:
\begin{enumerate}
\item
Microsoft Windows Event Logs collected by a centralized log management system;
\item
Logs in a well-documented format sent via \texttt{syslog}, \texttt{syslog-ng}\oxford{} or \texttt{syslog-reliable} network protocols to a centralized log management system;
\item
Logs stored in an \gls{ansi} \gls{sql} database that itself generates audit logs in compliance with the requirements of this document; and
\item
Other open logging mechanisms supporting the above requirements including those based on CheckPoint OpSec, ArcSight CEF, and \gls{idmef}.
\end{enumerate}
\CommonPolicyCompliance
\section{Related Standards, Policies\oxford{} and Processes}
None.
\section{Definitions and Terms}
None.
\CommonRevisionHistory