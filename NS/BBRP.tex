\chapter{Bluetooth Baseline Requirements Policy}
\CommonIntroduction
\section{Overview}
Bluetooth enabled devices are exploding on the Internet at an astonishing rate\oldnew{. At}{ and} the range of connectivity has increased substantially.  
Insecure Bluetooth connections can introduce a number of potential serious security issues.  
Hence, there is a need for a minimum standard for connecting Bluetooth enable\ins{d} devices.
\section{Purpose}
The purpose of this policy is to provide a minimum baseline standard for connecting Bluetooth enabled devices to the \CompanyName{} network or \CompanyName{} owned devices.   
The intent of the minimum standard is to ensure sufficient protection \gls{pii} and confidential \CompanyName{} data.
\section{Scope}
This policy applies to any Bluetooth enabled device that is connected to \CompanyName{} network or owned devices. 
\section{Policy}
\subsection{Version}
No Bluetooth Device shall be deployed on \CompanyName{} equipment that does not meet a minimum of Bluetooth v2.1 specifications without written authorization from the \gls{infosec} Team.  
Any Bluetooth equipment purchased prior to this policy must comply with all parts of this policy except the Bluetooth version specifications.
\subsection{Pins and Pairing}
When pairing your Bluetooth unit to your Bluetooth enabled equipment (\ie phone, laptop, \etc), ensure that you are not in a public area where you \gls{pin} can be compromised. 
If your Bluetooth enabled equipment asks for you to enter your pin after you have initially paired it, you must refuse the pairing request and report it to \gls{infosec}, through your Help Desk, immediately.  
\subsection{Device Security Settings}
\begin{itemize}
\item
All Bluetooth devices shall employ \q{security mode 3} which encrypts traffic in both directions, between your Bluetooth Device and its paired equipment.
\item
Use a minimum \gls{pin} length of 8.  
A longer \gls{pin} provides more security.
\item
Switch the Bluetooth device to use the hidden mode (non-discoverable)
\item
Only activate Bluetooth only when it is needed.
\item
Ensure device firmware is up-to-date. 
\end{itemize}
\subsection{Security Audits}
The \gls{infosec} Team may perform random audits to ensure compliancy with this policy.  
In the process of performing such audits, \gls{infosec} Team members shall not eavesdrop on any phone conversation.

\subsection{Unauthorized Use}
The following is a list of unauthorized uses of \CompanyName{}-owned Bluetooth devices:
\begin{itemize}
\item
Eavesdropping, device ID spoofing, \gls{dos} attacks, or any form of attacking other Bluetooth enabled devices.
\item
Using \CompanyName{}-owned Bluetooth equipment on non-\CompanyName{}-owned Bluetooth enabled devices.
\item
Unauthorized modification of Bluetooth devices for any purpose.
\end{itemize}
\subsection{User Responsibilities}
\begin{itemize}
\item
It is the Bluetooth user's responsibility to comply with this policy. 
\item
Bluetooth mode must be turned off when not in use.
\item
\gls{pii} and/or \CompanyName{} Confidential or Sensitive data must not be transmitted or stored on Bluetooth enabled devices. 
\item
Bluetooth users must only access \CompanyName{} information systems using approved Bluetooth device hardware, software, solutions\oxford{} and connections. 
\item
Bluetooth device hardware, software, solutions\oxford{} and connections that do not meet the standards of this policy shall not be authorized for deployment. 
\item
Bluetooth users must act appropriately to protect information, network access, passwords, cryptographic keys\oxford{} and Bluetooth equipment. 
\item
Bluetooth users are required to report any misuse, loss\oxford{} or theft of Bluetooth devices or systems immediately to \gls{infosec}.
\end{itemize}
\CommonPolicyCompliance
\section{Related Standards, Policies\oxford{} and Processes}
None.
\section{Definitions and Terms}
None.
\CommonRevisionHistory