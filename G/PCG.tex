\chapter{Password Construction Guidelines}\label{G:PCG}
\CommonIntroduction
\section{Overview}
Passwords are a critical component of information security.  
Passwords serve to protect user accounts; however, a poorly constructed password may result in the compromise of individual systems, data, or \oldnew{the Cisco network}{networks}.  
This guideline provides best practices for creating secure passwords.
\section{Purpose}
The purpose of this guidelines is to provide best practices for the created of strong passwords.
\section{Scope}
This guideline applies to employees, contractors, consultants, temporary and other workers\oldnew{ at Cisco}{}, including all personnel affiliated with third parties.  
This guideline applies to all passwords including but not limited to user-level accounts, system-level accounts, web accounts, \email{} accounts, screen\oldnew{ }{}saver protection, voicemail, and local router logins.
\section{Statement of Guidelines}
All passwords should meet or exceed the following guidelines.

Strong passwords have the following characteristics:
\begin{itemize}
\item
Contain at least 12 alphanumeric characters.
\item
Contain both upper and lower case letters. 
\item
Contain at least one number (for example, \texttt{0}-\texttt{9}).
\item
Contain at least one special character (for example,\texttt{!\$\%\^{}\&*()\_+|\~{}-=\textbackslash\{\}`\{\}[]:";'<>?,/}).%TODO
\end{itemize}
Poor, or weak, passwords have the following characteristics: 
\begin{itemize}
\item
Contain less than eight characters.
\item
Can be found in a dictionary, including foreign language, or exist in a language slang, dialect\oxford{} or jargon.
\item
Contain personal information such as birthdates, addresses, phone numbers, or names of family members, pets, friends, and fantasy characters.
\item
Contain work-related information such as building names, system commands, sites, companies, hardware, or software.
\item
Contain number patterns such as \texttt{aaabbb}, \texttt{qwerty}, \texttt{zyxwvuts}, or \texttt{123321}.
\item
Contain common words spelled backward, or preceded or followed by a number (for example, \texttt{terces}, \texttt{secret1} or \texttt{1secret}).
\item
Are some version of \q{\texttt{Welcome123}}\oldnew{}{,} \q{\texttt{Password123}}\oldnew{}{\oxford{} or} \q{\texttt{Changeme123}}.
\end{itemize}
You should never write down a password.  
Instead, try to create passwords that you can remember easily.  
One way to do this is create a password based on a song title, affirmation, or other phrase.  
For example, the phrase, \q{This \oldnew{M}{m}ay \oldnew{B}{b}e \oldnew{O}{o}ne \oldnew{W}{w}ay \oldnew{T}{t}o \oldnew{R}{r}emember} could become the password \texttt{TmB1w2R!} or another variation.

(NOTE: Do not use either of these examples as passwords!)

\subsection{Passphrases}
Passphrases generally are used for public/private key authentication.  
A public/private key system defines a mathematical relationship between the public key that is known by all, and the private key, that is known only to the user.  
Without the passphrase to unlock the private key, the user cannot gain access.

A passphrase is similar to a password in use; however, it is relatively long and constructed of multiple words, which provides greater security against dictionary attacks.  
Strong passphrases should follow the general password construction guidelines to include upper and lowercase letters, numbers, and special characters (for example, \texttt{TheTrafficOnThe101Was*\&!\$ThisMorning!}).

\CommonPolicyCompliance
\section{Related Standards, Policies\oxford{} and Processes}
None.
\section{Definitions and Terms}
None.
\section{Revision History}
\begin{tabular}{|p{1.25in}|p{1.25in}|p{3in}|}
\hline
	Date of Change&
	Responsible&
	Summary of Change\\
\hline
	June 2014&
	SANS Policy Team&
	Separated out from the \textsl{Password Policy} and converted to new format.\\
\hline
	Dec. 2016---Jan. 2017&
	\xio{}&
	Conversion to \LaTeX{}; removed references to Cisco.\\
\hline
	 &
	 &
	 \\
\hline
\end{tabular}
