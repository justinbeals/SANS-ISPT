\chapter{\Email{} Policy}
\CommonIntroduction
\section{Overview}
\oldnew{Electronic email}{\Email{}} is pervasively used in almost all industry verticals and is often the primary communication and awareness method within an organization.  
At the same time, misuse of \email{} can post many legal, privacy\oxford{} and security risks\oldnew{, t}{.  
T}hus it\oldnew{'}{ i}s important for users to understand the appropriate use of electronic communications. 
\section{Purpose}
The purpose of this \email{} policy is to ensure the proper use of \CompanyName{} \email{} system and make users aware of what \CompanyName{} deems as acceptable and unacceptable use of its \email{} system.  
This policy outlines the minimum requirements for use of \email{} within \CompanyName{} Network. 
\section{Scope}
This policy covers appropriate use of any \email{} sent from a \CompanyName{} \email{} address and applies to all employees, vendors, and agents operating on behalf of \CompanyName{}.
\section{Policy}
\begin{enumerate}
\item 
All use of \email{} must be consistent with \CompanyName{} policies and procedures of ethical conduct, safety, compliance with applicable laws\oxford{} and proper business practices.
\item 
\CompanyName{} \email{} account should be used primarily for \CompanyName{} business-related purposes; personal communication is permitted on a limited basis, but non-\CompanyName{} related commercial uses are prohibited.
\item 
All \CompanyName{} data contained within an \email{} message or an attachment must be secured according to the Data Protection Standard.%TODO Data Protection Standards
\item 
\Email{} should be retained only if it qualifies as a \CompanyName{} business record.  
\Email{} is a \CompanyName{} business record if there exists a legitimate and ongoing business reason to preserve the information contained in the \email{}.
\item 
\Email{} that is identified as a \CompanyName{} business record shall be retained according to \CompanyName{} Record Retention Schedule. %TODO Data Retention Schedule
\item 
The \CompanyName{} \email{} system shall not to be used for the creation or distribution of any disruptive or offensive messages, including offensive comments about race, gender, hair color, disabilities, age, sexual orientation, pornography, religious beliefs and practice, political beliefs, or national origin.  
Employees who receive any \email{}s with this content from any \CompanyName{} employee should report the matter to their supervisor immediately.
\item 
Users are prohibited from automatically forwarding \CompanyName{} \email{} to a third party \email{} system \del{(noted in 4.8 below)}.  
Individual messages which are forwarded by the user must not contain \CompanyName{} confidential or above information.
\item \label{G:EmP:P:1}
Users are prohibited from using third-party \email{} systems and storage servers such as Google, Yahoo, and MSN/Hotmail/Outlook, \etc, to conduct \CompanyName{} business, to create or memorialize any binding transactions, or to store or retain \email{} on behalf of \CompanyName{}.  
Such communications and transactions should be conducted through proper channels using \CompanyName{}-approved documentation. 
\item 
Using a reasonable amount of \CompanyName{} resources for personal \email{}s is acceptable, but non-work related \email{} shall be saved in a separate folder from work related \email{}.  
Sending chain letters or joke \email{}s from a \CompanyName{} \email{} account is prohibited.
\item 
\CompanyName{} employees shall have no expectation of privacy in anything they store, send\oxford{} or receive on the company's \email{} system.
\item 
\CompanyName{} may monitor messages without prior notice.  
\CompanyName{} is not obliged to monitor \email{} messages.
\end{enumerate}
\CommonPolicyCompliance
\section{Related Standards, Policies\oxford{} and Processes}
\begin{itemize}
\item {Data Protection Standard}%TODO 
\end{itemize}
\section{Definitions and Terms}
None.
\section{Revision History}
\begin{tabular}{|p{1.25in}|l|p{3in}|}
\hline
	Date of Change&
	Responsible&
	Summary of Change\\
\hline
	Dec 2013&
	SANS Policy Team&
	Updated and converted to new format.\\
\hline
	Dec. 2016\newline{}Jan. 2017&
	\xio{}&
	Conversion to \LaTeX{}.\\
\hline
	 &
	 &
	 \\
\hline
\end{tabular}
