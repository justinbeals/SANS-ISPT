\chapter{Password Protection Policy}\label{G:PPP}
\CommonIntroduction
\section{Overview}
Passwords are an important aspect of computer security.  
A poorly chosen password may result in unauthorized access and/or exploitation of \CompanyName{}'s resources.  
All users, including contractors and vendors with access to \CompanyName{} systems, are responsible for taking the appropriate steps, as outlined below, to select and secure their passwords.
\section{Purpose}
The purpose of this policy is to establish a standard for creation of strong passwords, the protection of those passwords\oxford{} and the frequency of change.
\section{Scope}
The scope of this policy includes all personnel who have or are responsible for an account (or any form of access that supports or requires a password) on any system that resides at any \CompanyName{} facility, has access to the \CompanyName{} network\oxford{} or stores any non-public \CompanyName{} information.
\section{Policy}
\subsection{Password Creation}
\begin{enumerate}
\item
All user-level and system-level passwords must conform to the \hyperref[G:PCG]{Password Construction Guidelines}.
\item
Users must not use the same password for \CompanyName{} accounts as for other non-\CompanyName{} access (for example, personal ISP account, option trading, benefits\oxford{} and so on).
\item
Where possible, users must not use the same password for various \CompanyName{} access needs.
\item
User accounts that have system-level privileges granted through group memberships or programs such as \texttt{sudo} must have a unique password from all other accounts held by that user to access system-level privileges.
\item
Where \gls{snmp} is used, the community strings must be defined as something other than the standard defaults of public, private\oxford{} and system \oldnew{and}{, and passwords} must be different from the passwords used to log in interactively.  
\gls{snmp} community strings must meet password construction guidelines.
\end{enumerate}
\subsection{Password Change}
\begin{enumerate}
\item
All system-level passwords (for example, \texttt{root}, enable, NT admin, application administration accounts\oxford{} and so on) must be changed on at least a quarterly basis.
\item
All user-level passwords (for example, \email{}, web, desktop computer\oxford{} and so on) must be changed at least every six months.  
The recommended change interval is every four months.
\item
Password cracking or guessing may be performed on a periodic or random basis by the \gls{infosec} team or its delegates.  
If a password is guessed or cracked during one of these scans, the user will be required to change it \oldnew{to be in}{in} compliance with the \hyperref[G:PCG]{Password Construction Guidelines}.
\end{enumerate}
\subsection{Password Protection}
\begin{enumerate}
\item
Passwords must not be shared with anyone.  
All passwords are to be treated as sensitive, Confidential \CompanyName{} information.  
Corporate Information Security recognizes that legacy applications do not support proxy systems in place.  
Please refer to the technical reference for additional details.
\item
Passwords must not be inserted into \email{} messages, Alliance cases\oxford{} or other forms of electronic communication.
\item
Passwords must not be revealed over the phone to anyone.
\item
Do not reveal a password on questionnaires or security forms.
\item
Do not hint at the format of a password (for example, \q{my family name}).
\item
Do not share \CompanyName{} passwords with anyone, including administrative assistants, secretaries, managers, co-workers while on vacation, and family members.
\item
Do not write passwords down and store them anywhere in your office.  
Do not store passwords in a file on a computer system or mobile devices (phone, tablet) without encryption.
\item
Do not use the \q{Remember Password} feature of applications (for example, web browsers).
\item
Any user suspecting that his/her password may have been compromised must report the incident and change all passwords.
\end{enumerate}
\subsection{Application Development}
Application developers must ensure that their programs contain the following security precautions:
\begin{enumerate}
\item
Applications must support authentication of individual users, not groups.
\item
Applications must not store passwords in clear text or in any easily reversible form.
\item
Applications must not transmit passwords in clear text over the network.
\item
Applications must provide for some sort of role management, such that one user can take over the functions of another without having to know the other's password.
\end{enumerate}
\subsection{Use of Passwords and Passphrases}
Passphrases are generally used for public/private key authentication.  
A public/private key system defines a mathematical relationship between the public key that is known by all, and the private key, that is known only to the user.  
Without the passphrase to \q{unlock} the private key, the user cannot gain access.

Passphrases are not the same as passwords.  
A passphrase is a longer version of a password and is, therefore, more secure.  
A passphrase is typically composed of multiple words.  
Because of this, a passphrase is more secure against \q{dictionary attacks}.

A good passphrase is relatively long and contains a combination of upper and lowercase letters and numeric and punctuation characters.  
An example of a good passphrase\oldnew{:}{ is} \q{\texttt{The*?\asciihash{}>\asciihash{}@Traf\/f\/icOnThe101Was*\asciiampersand{}\asciihash{}!\asciihash{}ThisMorning}}.

All of the rules above that apply to passwords apply to passphrases.
\CommonPolicyCompliance

\section{Related Standards, Policies\oxford{} and Processes}
\begin{itemize}
\item \hyperref[G:PCG]{Password Construction Guidelines}
\end{itemize}
\CommonDefinitionsAndTerms
\begin{itemize}
\item \acrfull{snmp}
\end{itemize}
\CommonRevisionHistory
