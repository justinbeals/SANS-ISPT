%	Transcriber's Note:
%	This document was poorly-worded, unsurprising given its relative novelty compared to other policy templates in this series.  It was also inconsistent with its sibling policy templates.  Amendments have been made.
\chapter{Data Breach Response Policy}
\setcounter{section}{0}
Created by or for the SANS Institute.  
Feel free to modify or use for your organization.  
If you have a policy to contribute, please send e-mail to \href{mailto:stephen@sans.edu}{stephen\at{}sans.edu}
\section{Purpose}
The purpose of the policy is to establish the goals and the vision for the breach response process.  
This policy will clearly define to whom it applies and under what circumstances, and it will include the definition of a breach, staff roles and responsibilities, standards and metrics (\eg, to enable prioritization of the incidents), as well as reporting, remediation, and feedback mechanisms.  
The policy shall be well publicized and made easily available to all personnel whose duties involve data privacy and security protection.

\OrganizationName{} \gls{infosec}'s intentions for publishing a data breach response policy are to focus significant attention on data security and data security breaches and how \OrganizationName{}'s established culture of openness, trust\oxford{} and integrity should respond to such activity.  
\OrganizationName{} \gls{infosec} is committed to protecting \OrganizationName{}'s employees, partners and the company from illegal or damaging actions by individuals, either knowingly or unknowingly. 

\subsection{Background}
This policy mandates that any individual who suspects that a theft, breach\oxford{} or exposure of \OrganizationName{} protected\del{ data} or\del{ \OrganizationName{}} sensitive data has occurred must immediately provide a description of what occurred:
\begin{itemize}
\item
via \email{} to \href{maito:helpdesk@example.com}{HelpDesk\at{}Example.com},% CHANGE ME
\item
by calling +1~(710)~555-1212,%CHANGE ME
\item
or through the use of the help desk reporting web page at \href{http://example.com}{http://Example.com}.% CHANGE ME
\end{itemize}
This \email{} address, phone number, and web page are monitored by the \OrganizationName{}'s \gls{infosec} administrator.  
This team will investigate all reported thefts, data breaches\oxford{} and exposures to confirm if a theft, breach\oxford{} or exposure has occurred.  
If a theft, breach\oxford{} or exposure has occurred, the \gls{infosec} administrator will follow the appropriate procedure in place.

\section{Scope}
This policy applies to all whom collect, access, maintain, distribute, process, protect, store, use, transmit, dispose of, or otherwise handle \gls{pii} or \gls{phi} of \OrganizationName{} members.  
Any agreements with vendors will contain language similar that protects the fund.

\section{Policy}
\subsection{Confirmed theft, breach\oxford{} or exposure of \OrganizationName{} protected or sensitive data}
As soon as a theft, data breach\oxford{} or exposure containing \OrganizationName{} protected \del{data }or \del{\OrganizationName{} }sensitive data is identified, the process of removing all access to that resource will begin.

The Executive Director will chair an incident response team to handle the breach or exposure.  
The team will include members from:
\begin{itemize}
\item
\gls{it} Infrastructure
\item
\gls{it} Applications
\item
Finance (if applicable)
\item
Legal
\item
Communications
\item
Member Services (if Member data is affected)
\item
Human Resources
\item
The affected unit or department that uses the involved system or output, or whose data may have been breached or exposed
\item
Additional departments based on the data type involved
\item
Additional individuals as deemed necessary by the Executive Director
\end{itemize}

\subsubsection{Confirmed theft, breach\oxford{} or exposure of \OrganizationName{} data}
The Executive Director will be notified of the theft, \ins{data} breach\oxford{} or exposure.  
\gls{it}, along with the designated forensic team, will analyze the breach or exposure to determine the root cause.  

\subsubsection{Work with Forensic Investigators}
As provided by \OrganizationName{} cyber insurance, the insurer will need to provide access to forensic investigators and experts that will determine:
\begin{itemize}
\item
how the breach or exposure occurred
\item
the type(s) of data involved
\item
the number of internal/external individuals and/or organizations impacted
\item
and analyze the breach or exposure to determine the root cause.
\end{itemize}

\subsubsection{Develop a Communication Plan}
Work with \OrganizationName{} communications, legal and human resource departments to decide how to communicate the breach to\del{: a)} internal employees,\del{ b)} the public, and\del{ c)} those directly affected.

\subsection{Ownership and Responsibilities}
\subsubsection{Roles and Responsibilities}
\paragraph{Sponsors} are those members of the \OrganizationName{} community that have primary responsibility for maintaining any particular information resource.  
Sponsors may be designated by any \OrganizationName{} Executive in connection with their administrative responsibilities, or by the actual sponsorship, collection, development, or storage of information.
\paragraph{Information Security Administrator} is that member of the \OrganizationName{} community, designated by the Executive Director or the Director, \gls{it} Infrastructure, who provides administrative support for the implementation, oversight and coordination of security procedures and systems with respect to specific information resources in consultation with the relevant Sponsors.
\paragraph{Users} include virtually all members of the \OrganizationName{} community to the extent they have authorized access to information resources, and may include staff, trustees, contractors, consultants, interns, temporary employees and volunteers.
\paragraph{Incident Response Team} shall be chaired by Executive Management and shall include, but will not be limited to, the following departments or their representatives:
\begin{itemize}
\item
\gls{it} Infrastructure
\item
\gls{it} Application Security
\item
Communications
\item
Legal
\item
Management
\item
Financial Services
\item
Member Services
\item
Human Resources
\end{itemize}

\section{Enforcement}
Any \OrganizationName{} personnel found in violation of this policy may be subject to disciplinary action, up to and including termination of employment.  
Any third party partner company found in violation may have their network connection terminated. 

\section{Definitions}\label{G:DBRP:D}
\begin{itemize}
\item{Encryption or encrypted data}
The most effective way to achieve data security.  
To read an encrypted file, you must have access to a secret key or password that enables you to decrypt it.  
Unencrypted data is called plain text.
\item{Plain text}
Unencrypted data.
\item{Hacker}
A slang term for a computer enthusiast, \ie{}, a person who enjoys learning programming languages and computer systems and can often be considered an expert on the subject(s).
\item{Protected Health Information}\label{G:DBRP:D:PHI}
Under US law is any information about health status, provision of health care, or payment for health care that is created or collected by a \q{covered entity} (or a business associate of a covered entity), and can be linked to a specific individual.
\item{Personally Identifiable Information}\label{G:DBRP:D:PII}
Any data that could potentially identify a specific individual.  
Any information that can be used to distinguish one person from another and\ins{/or} can be used \oldnew{for}{to} de-anonymiz\oldnew{ing}{e} anonymous data\del{ can be considered}
\item{Protected data}
See \gls{phi} and \gls{pii}, \see{G:DBRP:D:PHI}.
\item{Information Resource}
The data and information assets of an organization, department\oxford{} or unit.
\item{Safeguards}
Countermeasures; controls put in place to avoid, detect, counteract, or minimize security risks to physical property, information, computer systems, or other assets.  
Safeguards help to reduce the risk of damage or loss by stopping, deterring\oxford{} or slowing down an attack against an asset.
\item{Sensitive Data}
Data that is encrypted or in plain text, and contains \gls{pii} or \gls{phi}.  
\end{itemize}

\section{Revision History}
\begin{tabular}{|p{0.50in}|p{1.25in}|p{1.25in}|p{2.50in}|}
\hline
	Version&
	Date of Revision&
	Author&
	Description of Changes\\
\hline
	1.0.1&
	2017-01-05&
	\xio{}&
	Conversion to \LaTeX{}.\\
\hline
	1.0&
	August 17, 2016&
	SANS Institute&
	Initial Version\\
\hline
	1.0&
	&
	&
	\\
\hline
\end{tabular}