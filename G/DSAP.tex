\chapter{Digital Signature Acceptance Policy}
\CommonIntroduction
\section{Overview}
See \see{g:dsap:p}.
\section{Purpose}\label{g:dsap:p}
The purpose of this policy is to provide guidance on when digital signatures are considered accepted means of validating the identity of a signer in \CompanyName{} electronic documents and correspondence, and thus a substitute for traditional \q{wet} signatures, within the organization.  
Because communication has become primarily electronic, the goal is to reduce confusion about when a digital signature is trusted.
\section{Scope}
This policy applies to all \CompanyName{} employees and affiliates.  

This policy applies to all \CompanyName{} employees, contractors, and other agents conducting \CompanyName{} business with a \CompanyName{}-provided digital key pair.  
This policy applies only to intra-organization digitally signed documents and correspondence and not to electronic materials sent to or received from non-\CompanyName{} affiliated persons or organizations.
\section{Policy}
A digital signature is an acceptable substitute for a wet signature on any intra-organization document or correspondence, with the exception of those noted on the site of the \gls{cfo} on the organization's intranet\oldnew{: \textsc{CFO's Office URL goes here}}{.}

The \gls{cfo}'s office will maintain an organization-wide list of the types of documents and correspondence that are not covered by this policy.

Digital signatures must apply to individuals only.  
Digital signatures for roles, positions, or titles (\eg{} the \gls{cfo}) are not considered valid.
\subsection{Responsibilities}
Digital signature acceptance requires specific action on both the part of the employee signing the document or correspondence (hereafter the \oldnew{\textit{signer}}{\q{signer}}, and the employee receiving/reading the document or correspondence (hereafter the \oldnew{\textit{recipient}}{\q{recipient}}).
\subsection{Signer Responsibilities}
\begin{enumerate}
\item 
Signers must obtain a signing key pair from \oldnew{\textsc{Company Name identity management group}}{\CompanyName{} \IdentityManagementGroup{}}.  This key pair will be generated using \CompanyName{}'s \gls{pki} and the public key will be signed by the \CompanyName{}'s \gls{ca}, \CertificateAuthority{}.
\item 
Signers must sign documents and correspondence using software approved by \CompanyName{} \gls{it} organization.
\item 
Signers must protect their private key and keep it secret.
\item 
If a signer believes that the signer's private key was stolen or otherwise compromised, the signer must contact \CompanyName{} \IdentityManagementGroup{} immediately to have the signer's digital key pair revoked.
\end{enumerate}
\subsection{Recipient Responsibilities}
\begin{enumerate}
\item 
Recipients must read documents and correspondence using software approved by \CompanyName{} \gls{it} department.
\item 
Recipients must verify that the signer's public key was signed by the \CompanyName{}'s \gls{ca}, \CertificateAuthority{}, by viewing the details about the signed key using the software they are using to read the document or correspondence.
\item 
If the signer's digital signature does not appear valid, the recipient must not trust the source of the document or correspondence.
\item 
If a recipient believes that a digital signature has been abused, the recipient must report the recipient's concern to \CompanyName{} \IdentityManagementGroup{}.
\end{enumerate}
\CommonPolicyCompliance
\section{Related Standards, Policies and Processes}
None.
\section{References}
Note that these references were used only as guidance in the creation of this policy template.  
We highly recommend that you consult with your organization's legal counsel, since there may be federal, state, or local regulations to which you must comply.  
Any other \gls{pki}-related policies your organization has may also be cited here.
\begin{itemize}
\item
\href{http://www.abanet.org/scitech/ec/isc/dsgfree.html}{American Bar Association (ABA) Digital Signature Guidelines}\newline\url{http://www.abanet.org/scitech/ec/isc/dsgfree.html}
\item
\href{http://mn.gov/oet/policies-and-standards/business/policy-pages/standard_digital_signature.jsp}{Minnesota State Agency Digital Signature Implementation and Use}\newline\url{http://mn.gov/oet/policies-and-standards/business/policy-pages/standard_digital_signature.jsp}
\item
\href{https://www.revisor.leg.state.mn.us/statutes/?id=325K&view=chapter-stat.325K.001}{Minnesota Electronic Authentication Act}\newline\url{https://www.revisor.leg.state.mn.us/statutes/?id=325K&view=chapter-stat.325K.001}
\item
\href{http://mesa.cabq.gov/policy.nsf/WebApprovedX/4D4D4667D0A7953A87256E7B004F6720?OpenDocument}{City of Albuquerque E-Mail Encryption / Digital Signature Policy}\newline\url{http://mesa.cabq.gov/policy.nsf/WebApprovedX/4D4D4667D0A7953A87256E7B004F6720?OpenDocument}
\item
\href{http://law.justia.com/westvirginia/codes/39a/wvc39a-3-2.html}{West Virginia Code \S{}39A-3-2:  Acceptance of electronic signature by governmental entities in satisfaction of signature requirement.}\newline\url{http://law.justia.com/westvirginia/codes/39a/wvc39a-3-2.html} 
\end{itemize}
\section{Definitions and Terms}
None.
\CommonRevisionHistory