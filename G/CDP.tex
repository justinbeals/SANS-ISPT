\chapter{Clean Desk Policy}
\CommonIntroduction
\section{Overview}
A clean desk policy can be an import tool to ensure that all sensitive/confidential materials are removed from an end user workspace and locked away when the items are not in use or an employee leaves his/her workstation.  
It is one of the top strategies to utilize when trying to reduce the risk of security breaches in the workplace.  
Such a policy can also increase employee’s awareness about protecting sensitive information.
\section{Purpose}
The purpose for this policy is to establish the minimum requirements for maintaining a \q{clean desk} where sensitive/critical information about \del{our }employees, \del{our }intellectual property, \del{our }customers\oxford{} and \del{our }vendors is secure in locked areas and out of site.  
A clean desk policy is not only ISO 27001/17799 compliant, but it is also part of standard basic privacy controls.
\section{Scope}
This policy applies to all \CompanyName{} employees and affiliates.
\section{Policy}
\begin{enumerate}
\item
Employees are required to ensure that all sensitive/confidential information in hardcopy or electronic form is secure in their work area at the end of the day and when they are expected to be gone for an extended period. 
\item
Computer workstations must be locked when workspace is unoccupied.
\item
Computer workstations must be shut completely down at the end of the work day. 
\item
Any \oldnew{R}{r}estricted or \oldnew{S}{s}ensitive information must be removed from the desk and locked in a drawer when the desk is unoccupied and at the end of the work day.
\item
File cabinets containing \oldnew{R}{r}estricted or \oldnew{S}{s}ensitive information must be kept closed and locked when not in use or when not attended. 
\item
Keys used for access to \oldnew{R}{r}estricted or \oldnew{S}{s}ensitive information must not be left at an unattended desk. 
\item
Laptops must be either locked with a locking cable or locked away in a drawer. 
\item
Passwords may not be left on sticky notes posted on or under a computer, nor may they be left written down in an accessible location. 
\item
Printouts containing \oldnew{R}{r}estricted or \oldnew{S}{s}ensitive information should be immediately removed from the printer. 
\item
Upon disposal \oldnew{R}{r}estricted and/or \oldnew{S}{s}ensitive documents should be shredded in the official shredder bins or placed in the lock confidential disposal bins.
\item
Whiteboards containing \oldnew{R}{r}estricted and/or \oldnew{S}{s}ensitive information should be erased.
\item
Lock away portable computing devices such as laptops and tablets.
\item
Treat mass storage devices such as \cdrom{}, DVD or USB drives as sensitive and secure them in a locked drawer
\end{enumerate}
All printers and fax machines should be cleared of papers as soon as they are printed; this helps ensure that sensitive documents are not left in printer trays for the wrong person to pick up.
\CommonPolicyCompliance
\section{Related Standards, Policies\oxford{} and Processes}
None.
\section{Definitions and Terms}
None.
\CommonRevisionHistory
