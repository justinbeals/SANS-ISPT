%	Common bits of text that are repeated across documents

\newcommand{\CommonRevisionHistory}{%
\section{Revision History}
\begin{tabular}{|p{1.25in}|p{1.25in}|p{3in}|}
\hline
	Date of Change&
	Responsible&
	Summary of Change\\
\hline
	June 2014&
	SANS Policy Team&
	Updated and converted to new format.\\
\hline
	Dec. 2016---Jan. 2017&
	\xio{}&
	Conversion to \LaTeX{}.\\
\hline
	 &
	 &
	 \\
\hline
\end{tabular}
}

\newcommand{\CommonIntroduction}{%
\setcounter{section}{0}
\subsection*{Free Use Disclaimer}
This policy was created by or for the SANS Institute for the Internet community.  
All or parts of this policy can be freely used for your organization.  
There is no prior approval required. If you would like to contribute a new policy or updated version of this policy, please send email to \href{mailto:policy-resources@sans.org}{policy-resources\at{}sans.org}.
\subsection*{Things to Consider}
Please consult the Things to Consider \acrshort{faq} for additional guidelines and suggestions for personalizing the SANS policies for your organization.
\subsection*{Last Update Status}
Updated June 2014.
}

\newcommand{\CommonPolicyCompliance}{%
\section{Policy Compliance}
\subsection{Compliance Measurement}
The \gls{infosec} team will verify compliance to this policy through various methods, including but not limited to, periodic walk-throughs, video monitoring, business tool reports, internal and external audits, and feedback to the policy owner.
\subsection{Exceptions}
Any exception to the policy must be approved by the \gls{infosec} team in advance.
\subsection{Non-Compliance}
An employee found to have violated this policy may be subject to disciplinary action, up to and including termination of employment.
}

\newcommand{\CommonDefinitionsAndTerms}{%
\section{Definitions and Terms}
The following definition and terms can be found in the \href{https://www.sans.org/security-resources/glossary-of-terms/}{SANS Glossary} located at \url{https://www.sans.org/security-resources/glossary-of-terms/}
}